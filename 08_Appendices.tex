\begin{appendices}

\chapter{R-code - Exercise 1}

\begin{verbatim}
# PART 1.1
# sampling t-distributed observations
returns=rt(10000,10)

# Creating a criterion function with W=I
criterion_I<-function(para,x)
  {
    nu=para
    cond1=mean(x^4)-((6)/(nu-4)+3)*mean(x^2)^2
    cond2=mean(x^2)-(nu/(nu-2))
    # output=y^Ty
    output=matrix(c(cond1,cond2),1,2)
    return(output%*%t(output))
  }

# Creating a criterion function with the full W
criterion_W<-function(para,x)
  {
    nu=para
    cond1=mean(x^4)-((6)/(nu-4)+3)*mean(x^2)^2
    cond2=mean(x^2)-(nu/(nu-2))
    W=cov(cbind(x^2/sqrt(length(x)),x^4/sqrt(length(x))))
    output=matrix(c(cond1,cond2),1,2)
    return(output%*%solve(W)%*%t(output))
  }

# Setting a potential list of candidate estimates
para_list=seq(5,30,by=1)

\end{verbatim}
\newpage
\begin{verbatim}
#USING IDENTITY
# Computing the criterion function for each estimate...
output_1=c()
for (i in 1:length(para_list))
  {
    output_1=c(output_1,criterion_I(para_list[i],returns))
  }

#... and then graph them!
pdf(file="t-returns_criterion_(W=I).pdf")
plot(para_list,output_1,type="l",xlab="nu",ylab="Criterion",
main="W=I")
dev.off()

# Finding the minimum value
minimum_1=para_list[which(output_1==min(output_1))]
print(minimum_1)

# And charting the resulting density
pdf(file="t-returns_density_(W=I).pdf")
temp=density(scale(returns))
plot(temp, type="l", ylab="Density", 
main="t-returns density (W=I)")
lines(temp$x,dt(temp$x,5),col="red")
dev.off()


# USING COVARIANCE MATRIX
# Computing the criterion function for each estimate...
output_2=c()
for (i in 1:length(para_list))
{
  output_2=c(output_2,criterion_W(para_list[i],returns))
}

#... and then graph them!
pdf(file="t-returns_criterion_(W=Sigma^-1).pdf")
plot(para_list,output_2,type="l",xlab="nu",
     ylab="Criterion",main="W=Cov(x)")
dev.off()

# Finding the minimum value
minimum_2=para_list[which(output_2==min(output_2))]
print(minimum_2)

# And charting the resulting density
pdf(file="t-returns_density_(W=Sigma^-1).pdf")
temp=density(scale(returns))
plot(temp, type="l", ylab="Density",
main="t-returns density (W=Sigma^-1)")
lines(temp$x,dt(temp$x,5),col="red")
dev.off()
\end{verbatim}

\newpage
\begin{verbatim}
# PART 1.2
x=read.delim("sp.csv",sep=";",header=FALSE)
x=diff(as.matrix(log(x)),1)

# sampling t-distributed observations
returns=(x-mean(x))/sd(x)
       
# Creating a criterion function with W=I
criterion_I<-function(para,x)
 {
   nu=para
   cond1=mean(x^4)-((6)/(nu-4)+3)*mean(x^2)^2
   cond2=mean(x^2)-(nu/(nu-2))
   # output=y^Ty
   output=matrix(c(cond1,cond2),1,2)
   return(output%*%t(output))
 }
       
# Creating a criterion function with the full W
criterion_W<-function(para,x)
 {
   nu=para
   cond1=mean(x^4)-((6)/(nu-4)+3)*mean(x^2)^2
   cond2=mean(x^2)-(nu/(nu-2))
   W=cov(cbind(x^2/sqrt(length(x)),x^4/sqrt(length(x))))
   output=matrix(c(cond1,cond2),1,2)
   return(output%*%solve(W)%*%t(output))
 }
       
# Setting a potential list of candidate estimates
para_list=seq(5,30,by=1)
     
     
#USING IDENTITY MATRIX
# Computing the criterion function for each estimate...
output_3=c()
for (i in 1:length(para_list))
 {
   output_3=c(output_3,criterion_I(para_list[i],returns))
 }
       
#... and then graph them!
pdf(file="S&P500_returns_criterion_(W=I).pdf")
plot(para_list,output_3,type="l",xlab="nu",ylab="Criterion",
main="W=I")
dev.off()

# Finding the minimum value
minimum_3=para_list[which(output_3==min(output_3))]
print(minimum_3)
       
# And charting the resulting density
pdf(file="S&P500_returns_density_(W=I).pdf")
temp=density(scale(returns))
plot(temp, type="l", ylab="Density", 
main="S&P500 returns density (W=I)")
lines(temp$x,dt(temp$x,5),col="red")       
dev.off()


# USING COVARIANCE MATRIX
# Computing the criterion function for each estimate...
output_4=c()
for (i in 1:length(para_list))
 {
   output_4=c(output_4,criterion_W(para_list[i],returns))
 }
       
#... and then graph them!
pdf(file="S&P500_returns_criterion_(W=Sigma^-1).pdf")
plot(para_list,output_4,type="l",xlab="nu",ylab="Criterion",
main="W=Cov(x)")
dev.off()

# Finding the minimum value
minimum_4=para_list[which(output_4==min(output_4))]
print(minimum_4)

# And charting the resulting density
pdf(file="S&P500_returns_density_(W=Sigma^-1).pdf")
temp=density(scale(returns))
plot(temp, type="l", ylab="Density", 
main="S&P500 returns density (W=Sigma^-1)")
lines(temp$x,dt(temp$x,5),col="red")       
dev.off()

\end{verbatim}
\end{appendices}
