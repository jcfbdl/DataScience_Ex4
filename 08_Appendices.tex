\begin{appendices}

\chapter{R-File - Exercise 1}

\begin{verbatim}
# PART 1.1
# sampling t-distributed observations
returns=rt(10000,10)
# Creating a criterion function with W=I
criterion_I<-function(para,x) {
    nu=para
    cond1=mean(x^4)-((6)/(nu-4)+3)*mean(x^2)^2
    cond2=mean(x^2)-(nu/(nu-2))
    # output=y^Ty
    output=matrix(c(cond1,cond2),1,2)
    return(output%*%t(output))
  }

# Creating a criterion function with the full W
criterion_W<-function(para,x) {
    nu=para
    cond1=mean(x^4)-((6)/(nu-4)+3)*mean(x^2)^2
    cond2=mean(x^2)-(nu/(nu-2))
    W=cov(cbind(x^2,x^4))
    output=matrix(c(cond2,cond1),1,2)
    return(output%*%solve(W)%*%t(output))
  }

# Setting a potential list of candidate estimates
para_list=seq(5,30,by=1)

#USING IDENTITY
# Computing the criterion function for each estimate...
output_1=c()
for (i in 1:length(para_list)) {
    output_1=c(output_1,criterion_I(para_list[i],returns))
  }

# Finding the minimum value
minimum_1=para_list[which(output_1==min(output_1))]
print(minimum_1)

#... and then graph them!
pdf(file="t-returns_criterion_(W=I).pdf")
plot(para_list,output_1,type="l",xlab="nu",ylab="Criterion",
main="W=I")
abline(v=minimum_1, col="purple", lty=2)
abline(v=10,col="red")
legend("topright", c("Estimated", "True Value"),
col=c("purple","red"),lty=2:1, cex=1.0 )
dev.off()

# And charting the resulting density
pdf(file="t-returns_density_(W=I).pdf")
temp=density(scale(returns))
plot(temp, type="l", ylab="Density", main="t-returns density (W=I)")
lines(temp$x,dt(temp$x,minimum_1),col="red")
dev.off()

# USING COVARIANCE MATRIX
# Computing the criterion function for each estimate...
output_2=c()
for (i in 1:length(para_list)) {
  output_2=c(output_2,criterion_W(para_list[i],returns))
}

# Finding the minimum value
minimum_2=para_list[which(output_2==min(output_2))]
print(minimum_2)

#... and then graph them!
pdf(file="t-returns_criterion_(W=Sigma^-1).pdf")
plot(para_list,output_2,type="l",xlab="nu",
     ylab="Criterion",main="W=Cov(x)")
abline(v=minimum_2, col="purple",lty=2)
abline(v=10,col="red")
legend("topright", c("Estimated", "True Value"),
col=c("purple","red"),lty=2:1, cex=1.0 )
dev.off()

# And charting the resulting density
pdf(file="t-returns_density_(W=Sigma^-1).pdf")
temp=density(scale(returns))
plot(temp, type="l", ylab="Density", main="t-returns density
(W=Sigma^-1)")
lines(temp$x,dt(temp$x,minimum_2),col="red")
dev.off()
\end{verbatim}

\chapter{R-File - Exercise 2}

\begin{verbatim}
# PART 1.2
x=read.delim("sp.csv",sep=";",header=FALSE)
x=diff(as.matrix(log(x)),1)

# sampling t-distributed observations
returns=(x-mean(x))/sd(x)
       
# Creating a criterion function with W=I
criterion_I<-function(para,x) {
   nu=para
   cond1=mean(x^4)-((6)/(nu-4)+3)*mean(x^2)^2
   cond2=mean(x^2)-(nu/(nu-2))
   # output=y^Ty
   output=matrix(c(cond1,cond2),1,2)
   return(output%*%t(output))
 }
       
# Creating a criterion function with the full W
criterion_W<-function(para,x) {
   nu=para
   cond2=mean(x^2)-(nu/(nu-2))
   cond1=mean(x^4)-((6)/(nu-4)+3)*mean(x^2)^2
   W=cov(cbind(x^2,x^4))
   output=matrix(c(cond2,cond1),1,2)
   return(output%*%solve(W)%*%t(output))
 }

W=cov(cbind(x^2,x^4))
W
para_list=seq(5,30,by=1)

#USING IDENTITY MATRIX
# Computing the criterion function for each estimate...
output_3=c()
for (i in 1:length(para_list)) {
   output_3=c(output_3,criterion_I(para_list[i],returns))
 }
       
# Finding the minimum value
minimum_3=para_list[which(output_3==min(output_3))]
print(minimum_3)

#... and then graph them!
pdf(file="S&P500_returns_criterion_(W=I).pdf")
plot(para_list,output_3,type="l",xlab="nu",ylab="Criterion"
,main="W=I")
abline(v=minimum_3, col="purple",lty=2)
legend("bottomright", c("Estimated"),col=c("purple"),
lty=2, cex=1.0 )
dev.off()

# And charting the resulting density
pdf(file="S&P500_returns_density_(W=I).pdf")
temp=density(scale(returns))
plot(temp, type="l", ylab="Density", main="S&P500 returns
density (W=I)")
lines(temp$x,dt(temp$x,minimum_3),col="red")       
dev.off()

# USING COVARIANCE MATRIX
# Computing the criterion function for each estimate...
output_4=c()
for (i in 1:length(para_list)) {
   output_4=c(output_4,criterion_W(para_list[i],returns))
 }

# Finding the minimum value
minimum_4=para_list[which(output_4==min(output_4))]
print(minimum_4)

pdf(file="S&P500_returns_criterion_(W=Sigma^-1).pdf")
plot(para_list,output_4,type="l",xlab="nu",ylab="Criterion",
main="W=Cov(x)")
abline(v=minimum_4, col="purple",lty=2)
legend("top", c("Estimated"),col=c("purple"), lty=2, cex=1.0 )
dev.off()

pdf(file="S&P500_returns_density_(W=Sigma^-1).pdf")
temp=density(scale(returns))
plot(temp, type="l", ylab="Density", main="S&P500 returns
density (W=Sigma^-1)")
lines(temp$x,dt(temp$x,minimum_4),col="red")       
dev.off()
\end{verbatim}

\chapter{Plotting Derivatives}

\begin{verbatim}
####################################################################
# Marceau Plotting Stuff & Shit
####################################################################

# Preparing Data

setwd("/Users/marceau/Desktop/HEC_BA/MA1/DataScience_For_Finance/")

y=read.delim("SP500_data.csv",sep=";",header=FALSE)

y=y[,2]
y=diff(as.matrix(log(y)),1)
returns=(y-mean(y))/sd(y)

a=mean(returns^4)
b=mean(returns^2)

f<-expression((a-(6/(x-4)+3)*b^2)^2+(b-x/(x-2))^2)
ff<-D(f,"x")
fff<-D(ff,"x")

f
ff
fff

g<-function(x){(((a-(6/(x-4)+3)*b^2)^2+(b-x/(x-2))^2))}
gg<-function(x){(2 * (6/(x - 4)^2 * b^2 * (a - (6/(x - 4)
                  + 3) * b^2)) - 2 * ((1/(x - 2) - x/(x -
                  2)^2) * (b - x/(x - 2))))}
ggg<-function(x){(2 * (6/(x - 4)^2 * b^2 * (6/(x - 4)^2 *
                  b^2) - 6 * (2 * (x - 4))/((x - 4)^2)^2
                  * b^2 * (a - (6/(x - 4) + 3) * b^2)) +
                  2 * ((1/(x - 2) - x/(x - 2)^2) * (1/(x
                  - 2) - x / (x - 2)^2) + (1/(x - 2)^2 +
                  (1/(x - 2)^2 - x * (2 * (x - 2))/((x -
                  2)^2)^2)) * (b - x/(x - 2))))}

xlim_1 = 1
xlim_2 = 6
ylim_1 = -7000
ylim_2 = 14000
num = 1000
  
pdf(file="Derivatives.pdf")
plot(g, from = xlim_1, to = xlim_2, n = num,
      ylim = c(ylim_1,ylim_2),
      xlab = "Parameter Value", ylab = "Objective Function",
      col = "blue", lwd = 1,
      main = "Objective Function & Derivatives for W=I")
      legend("topright", c("Objective Function",
                           "First Derivative", "Second Derivative"),
      col=c("blue", "red", "green"), lty=1, cex=0.7 )
par(new = TRUE)
plot(gg, from = xlim_1, to = xlim_2, n = num,
      ylim = c(ylim_1,ylim_2),
      axes = FALSE, xlab = "", ylab = "",
      col = "red", lwd = 1)
par(new = TRUE)
plot(ggg, from = xlim_1, to = xlim_2, n = num,
      ylim = c(ylim_1,ylim_2),
      axes = FALSE, xlab = "", ylab = "",
      col = "green", lwd = 1)
abline(h=0, col="black",lty=1 , lwd = 0.5)
dev.off()
\end{verbatim}


\end{appendices}
