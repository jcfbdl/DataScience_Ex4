\newpage
\section{Create two functions in R and compute the GMM with the S\&P500 returns}

Adopting the exact same approach as previously, moment conditions are defined as follows:
\begin{equation*}
    Y=    
    \begin{bmatrix}[l]
    E[X^4]-E[X^2]^2(\frac{6}{\nu-4}+3)  \\
    E[X^2]-\frac{\nu}{\nu-2}
    \end{bmatrix}
\end{equation*}

The function will then return the scalar resulting from the quadratic form $Y^TWY$. However, in this particular case we used the S\&P500 sample of returns instead of randomly generated ones; the methodology will be exactly the same: minimizing the moment conditions with respect to $\nu$. The following quadratic form is solved for different parameter candidates and two weight matrices ($W=I$ and $W=\Sigma^{-1}$.
\begin{equation*}
    Y^TWY
\end{equation*}
\begin{equation*}
    W_1=
    \begin{bmatrix}
        1   &0 \\
        0   &1
    \end{bmatrix};\;\;\;\;\;
    W_2=\Sigma^{-1}=
    \begin{bmatrix}[l]
        \frac{1}{\sigma_{m_1,m_1}}    &\frac{1}{\sigma_{m_1,m_2}} \\
        \frac{1}{\sigma_{m_2,m_1}}    &\frac{1}{\sigma_{m_2,m_2}}
    \end{bmatrix}
\end{equation*}

Running the code for the two cases generate a vector containing the output of the minimization problem for every parameter $\nu$ employed. Recall that by the first moment condition $\nu>4$ therefore a parameter list of the following form is set: $\nu_i \in \{5:30\}$. The goal is to find the parameter $\nu$ for which the objective function, i.e. criterion, is minimized, the resulting plots are listed on the following page.
\bigskip\par
The data used in the exercise gives an estimated $\nu$ of:
\begin{equation*}
    \begin{cases}
    \widehat{\nu}_{GMM_{1}}=5 \;\;\text{using W=I}\\
    \widehat{\nu}_{GMM_{2}}=30 \;\;\text{using W=}\Sigma^{-1}
\end{cases}
\end{equation*}

\subsection{W=I}
Why is the convexity apparently disappeared ? The criterion function has a minimum at $\nu_1=5$ and trying to run the code while changing the candidates list to $\nu_i \in \{5:30\}$ won't change the result: the minimum will always be at the first candidate tested. Chapter 2 aims to provide an explanation to this phenomenon.

\subsection{W=$\Sigma^{-1}$}
At first sight it seems correct to have convexity, the problem here is that we have a minimum for the exactly last candidate. Expanding the list of candidates from $\nu_i \in \{5:30\}$ to $\nu_i \in \{5:1000\}$ won't change the outcome, in this case the minimum would be at $\nu=1000$. To understand the phenomenon, the covariance matrix has to be analyzed.
\begin{equation*}
    W=\Sigma=
    \begin{bmatrix}[l]
        31.84035    &8250.907 \\
        8250.90722  &2839215.102
        \end{bmatrix}
\end{equation*}
Notice that the variance associated to the second moment condition is 89170.3483 times bigger than the variance associated to the second moment condition. The problem comes when we calculate the optimum function, in order to compute the output we have to invert $\Sigma$, therefore $\Sigma^{-1}$ is:
\begin{equation*}
    W=\Sigma=
    \begin{bmatrix}[l]
        \frac{1}{31.84035}    &\frac{1}{8250.907} \\
        \frac{1}{8250.90722}  &\frac{1}{2839215.102}
        \end{bmatrix}=
    \begin{bmatrix}[l]
        0.1271814035    &-3.695958*10^{-4} \\
        -0.0003695958    &1.426275*10^{-06}
    \end{bmatrix}
\end{equation*}





\newpage

\begin{figure}
    \centering
    \includegraphics[width=0.7\textwidth]{S&P500_returns_criterion_(W=I).pdf}
    \label{SP500_returns_criterion_I}
    \caption{Output distribution as a function of the candidates $\nu$ using $W=I$}
    \includegraphics[width=0.7\textwidth]{S&P500_returns_criterion_(W=Sigma^-1).pdf}
    \label{SP500_returns_criterion_W}
    \caption{Output distribution as a function of the candidates $\nu$ using $W=\Sigma^{-1}$}
\end{figure}

\subsection{Estimated vs. theoretical density functions}
Figure \ref{SP500_returns_density_I} compares the estimated density function based on the SP500 data using the $I$ as the weighting matrix with the theoretical Student t-distribution with 5 degrees of freedom, since we found that $df=5$ gave the estimated minimum of the distance function. Figure \ref{SP500_returns_density_W} compares the estimated density function based on the SP500 data using the covariance matrix between the two moment conditions as the weighting matrix with the theoretical Student t-distribution with 30 degrees of freedom, which again gave us the estimated minimum of our objective function. In both cases our estimate seems to be unbiased, centered around 0, but from the real dataset (black line) we get a taller and skinnier distribution than the theoretical distribution (red line). We could use statistical test, for example $\chi^2$ test to see whether there is significant difference between the two densities. 

\subsection{What do we deduce from the estimated $\nu$ value?}
We have found very different estimates of the parameter depending on the weighting matrix: $\nu=5$ when $W=I$ and $\nu=30$ when we use the inverse of the covariance matrix between the two moment conditions. In the first case we have a "naive" estimate where we put equal weights on the data no matter their accuracy and reliability. In fact, this is only the first step to get to our final estimate, so we can disregard this intermediate result. When using the inverse of the covariance matrix, we put more reasonable weights on the data that are more reliable and our estimate of $\nu=30$. Moreover, if we relax the assumption that $\nu\leq 30$, we would get an even higher estimate for the parameter. 

Since the Student t-distribution is derived from the Normal distribution and with $df>30$ it tends to the Normal distribution, we can deduce that our data is in fact normally distributed. Our data are log returns on a stock index, which is the average of the returns of several assets. Even if the log return on individual assets are not normally distributed, their average will follow the normal distribution by the central limit theorem. So we can conclude that log returns on the S\&P500 data follow a normal distribution.       


\begin{figure}
    \centering
    \includegraphics[width=0.7\textwidth]{S&P500_returns_density_(W=I).pdf}
    \label{SP500_returns_density_I}
    \caption{S\&P500 returns density using $W=I$}
    \includegraphics[width=0.7\textwidth]{S&P500_returns_density_(W=Sigma^-1).pdf}
    \label{SP500_returns_density_W}
    \caption{S\&P500 returns density using $W=\Sigma^{-1}$}
\end{figure}
